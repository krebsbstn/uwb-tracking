\chapter{Bluetooth Handling}
\label{chap:Bluetooth_Handling}
In this project, Bluetooth Low Energy (BLE) connectivity is utilized to enable the programming of new anchor values onto the tag device from a desktop application, while also facilitating the reception and visualization of the tag's current position. To achieve this, the project includes the "NimBLEDevice.h" header file, which is crucial for configuring and controlling the BLE functionality. It offers important functions and macros for configuring the BLE device, such as specifying the device's name, appearance, and advertising settings.
\vspace{4pt}
\newline
The structure and behavior of the BLE server used in this project are defined in the "ble-server.h" and "ble-server.cpp" files. These files play a pivotal role in managing the communication and interaction between the tag device and the desktop software, allowing for the seamless exchange of anchor values and real-time positioning data over the BLE connection. 
\vspace{4pt}
\newline
Since the data to be transmitted only affects the tag, it is the only device in the setup to host an BLE server. 

\section{Server structure}
\label{sec:Server_Structure}
In a Bluetooth Low Energy (BLE) server, services are discrete components that house related data and functionality, and each service is composed of one or more characteristics, each representing a specific aspect of that service. In this particular project, the tag device's server is configured with two distinct services: an input service and an output service.
\vspace{4pt}
\newline
The input service is designated for transferring data between the tag device and an external computer running UWB desktop software, enabling data input from the computer to the device, while the output service facilitates the opposite data flow, allowing the tag device to communicate data back to the computer. This separation of services streamlines the bidirectional data exchange process, effectively handling data transfer in both directions between the tag device and the external computer, enhancing the project's functionality and versatility.

\subsubsection{Input Service}
\label{subsub:Input_Service}
The input service serves as a crucial configuration tool for the tag device, allowing the adjustment of various parameters, particularly the positions of multiple anchors. This service is uniquely identified by the UUID "76847a0a-2748-4fda-bcd7-74425f0e4a10" and comprises two essential characteristics.
\vspace{4pt}
\newline
The \blueconsolas{DEVICE\_POSITION} characteristic is responsible for defining anchor positions, represented as strings that contain the x, y, and z coordinates. This characteristic acts as a means to input and update the spatial information of each anchor, facilitating precise location setup.
\vspace{4pt}
\newline
The \blueconsolas{SAVE\_CONFIG} characteristic is employed to initiate the saving process, which is responsible for persistently storing all updated anchor positions in the tag device's EEPROM (Electrically Erasable Programmable Read-Only Memory). This ensures that configuration changes to anchor positions are retained, even after power cycles, maintaining the device's accuracy and functionality.

\subsubsection{Output Service}
\label{subsub:Output_Service}
The output service plays a vital role in transmitting both the currently stored anchor positions and the tag device's own estimated position to the UWB configuration desktop software, which acts as the configuration device. This service is uniquely identified by the UUID "76847a0a-2748-4fda-bcd7-74425f0e4a20" and, similar to the input service, comprises two key characteristics.
\vspace{4pt}
\newline
The \blueconsolas{ANCHOR\_POSITIONS} characteristic is responsible for encapsulating all the presently saved anchor positions. This data is structured as a JSON file and serialized into a string. The process involves iterating through the locally stored landmarks and populating the JSON object with this information, along with their corresponding device IDs. This characteristic enables the transmission of anchor positions for configuration or reference.
\vspace{4pt}
\newline
The \blueconsolas{OWN\_POSITION} characteristic holds the tag device's own estimated X, Y, and Z coordinates. These coordinates are bundled into a string format and transmitted. This feature allows the tag device to relay its real-time positional information to the configuration device, which can be valuable for various location-based applications.

\section{BleServer Class}
\label{sec:BleServer_Class}
The BleServer class represents a Bluetooth server, responsible for initializing the Bluetooth device, managing services and characteristics, and handling data communication with connected devices. 
It organizes BLE services and characteristics using dynamic memory management, allowing for flexibility and extension. 
The class facilitates advertising, enables data reading and sending, and provides information about the number of connected devices.

\subsection{BleServer Variables}
\label{sub:BleServer_Variables}

\subsubsection{Private Variables}

\begin{itemize}
	\item \textbf{BLEServer *pServer}
	\newline
	pServer is a private member variable in the BleServer class, holding a pointer to a BLEServer object. 
	This object represents the Bluetooth server for managing BLE communication on an ESP32 device. 
	Throughout the class, pServer is utilized to create services, manage characteristics, and interact with the Bluetooth server, enabling the establishment of BLE communication.
	
	\item \textbf{BLEAdvertising *pAdvertising}
	\newline
	The pAdvertising variable is a pointer to an instance of the BLEAdvertising class. 
	This object is responsible for managing advertising settings and data for the Bluetooth server. 
	It is utilized during the initialization process to configure advertising intervals, associate service UUIDs, and start the advertising process. 

	\item \textbf{std::list<BLEService *> mServices}
	\newline
	The mServices variable is a linked list that holds pointers to instances of the BLEService class. 
	It is used to manage and store the Bluetooth services created during the initialization process. 
	By maintaining a list of services, the server can easily iterate through them for various operations, such as adding characteristics or starting the services. 
	
	\item \textbf{struct Characteristic}
	\newline
	he Characteristic struct encapsulates information about a Bluetooth characteristic, including its human-readable name, characteristic UUID, and descriptor UUID. 
	This struct is utilized within the BleServer class to organize and represent individual characteristics, contributing to the modular and well-structured design of the Bluetooth server. 
	By providing a clear structure for characteristic properties, it facilitates the creation and management of Bluetooth characteristics within the BLE server implementation.
	
	\item \textbf{struct Service}
	\newline
	The Service struct defines a Bluetooth service within the BleServer class, encapsulating a service UUID and an array of characteristics. 
	This struct aids in organizing and representing Bluetooth services, contributing to the modular architecture of the BLE server implementation. 
	By using the Service struct, the code achieves a clear and structured representation of Bluetooth services and their associated characteristics. 
	
	\item \textbf{const std::array<Service, 2> my\_services}
	\newline
	The my\_services constant is an array with a fixed size of 2, where each element is of type Service struct. 
	This array represents the services associated with the Bluetooth server and their respective characteristics. 
	It is initalized already with two services that are the inout and output services discussed in \ref{sec:Server_Structure}. 
	
\end{itemize}

\subsection{void init\_server()}
\label{sub:init_server}
The init\_server method initializes the Bluetooth server. 
First it creates a BLEDevice object of the NimBLE library naming it ESP32. 
Then it creates a server using the createServer method of BLEDevice and puts the pointer to this server in the pServer variable. 
After that the init\_services described in \ref{sub:init_services} method is called. 
Then an the pAdvertising pointer to the advertising object of the BLEDevice object is set using the getAdvertising method. 
The advertising is then configured by adding the UUIDs  of the input and output services using the addServiceUUID method and the advertising interval uding the setMinPreferred and setMaxPreferred methods. 
Subsequently the advertising is started by calling the BLEDevice method startAdvertising and a message is put on the serial monitor indicating that the server is now active and can be found by other Bluetooth devices nearby. 

\subsection{void init\_services()}
\label{sub:init_services}
The init\_services method initializes all Bluetooth services by creating instances of the BLEService class and associating them with their respective characteristics. 
It iterates through the my\_services array described in \ref{sub:BleServer_Variables} and creates a service for each element of type "Service" adding the respective characteristics for that service. 

\subsection{std::string read\_value(const std::string uuid)}
\label{sub:read_value}
The read\_value method reads and retrieves the current value of a specified Bluetooth characteristic identified by its UUID. 
It iterates through the list of services, locates the desired characteristic using its UUID, and returns the characteristic's value as a std::string. 
If the specified characteristic is not found, it prints a debug message to the Serial Monitor. 
This function facilitates the retrieval of information from specific BLE characteristics during the server's operation.

\subsection{void send\_value(std::string uuid, const std::string data)}
\label{sub:send_value}
The send\_value method updates the value of a specified Bluetooth characteristic identified by its UUID and notifies connected devices about the change. 
It locates the desired characteristic within the list of services, sets its value to the provided data, and triggers a notification to inform connected devices about the update. 
If the specified characteristic is not found, it prints a debug message to the Serial Monitor. 

\subsection{size\_t getConnectedCount()}
\label{sub:getConnectedCount}
The getConnectedCount method retrieves the number of devices currently connected to the Bluetooth server. 
It uses the getConnectedCount method of the BLEServer object to determine the count of connected devices. 

\subsection{void add\_Characteristic(BLEService *service, BleServer::Characteristic characteristic)}
\label{sub:add_Characteristic}
The add\_Characteristic function adds a new characteristic with a specified UUID to a given Bluetooth service that it both takes in as arguments. 
It utilizes the createCharacteristic method of the BLEService object to create a new characteristic and associates it with the provided UUID. 
Additionally, a descriptor is created for the characteristic, allowing for additional information to be attached. 

\section{BleConfigLoader Class}
\label{sec:BleConfigLoader}
The BleConfigLoader class serves as a fundamental component in the tag device, responsible for initializing and managing the server structure previously discussed in Section \ref{sec:Server_Structure}. During its initialization, it activates the server and periodically broadcasts its presence at intervals defined by the constants BLE\_MIN\_INTERVAL and BLE\_MAX\_INTERVAL in the ble-server.h file.
\vspace{4pt}
\newline
Within this class, there are methods for extracting anchor position information from incoming BLE characteristics, saving this data locally, and writing it to the device's EEPROM for persistent storage. Additionally, it includes methods for sending individual characteristics when called. For a more comprehensive understanding of these methods, please consult the DOXYGEN Documentation.
\vspace{4pt}
\newline
In the main.cpp file, an instance of the BleConfigLoader object is created, and its methods are invoked to enable the tag device's Bluetooth functionality. The actual invocation occurs within the BLE\_Task, as detailed in Section \ref{sec:Ble_Task}. This class encapsulates the core BLE configuration and communication logic, facilitating seamless operation and data exchange in the tag device.

\subsection{BleConfigLoader Variables}
\label{sub:BleConfigLoader_Variables}

\subsubsection{Private Variables}

\begin{itemize}
	\item \textbf{BleServer my\_server}
	\newline
	The my\_server object is an instance of the BleServer class referenced in \ref{sec:BleServer_Class}. 
	It is created in the BleConfigLoader class and serves as a Bluetooth server for managing BLE communication. 
	The my\_server object is used to initialize the BLE server, handle configuration settings, and facilitate communication between devices over Bluetooth Low Energy.
	
	\item \textbf{coordinate landmarkAddresses[NUM\_LANDMARKS]}
	\newline
	The landmarkAddresses is an array of coordinate structures in the BleConfigLoader class, with a size specified by NUM\_LANDMARKS. 
	Each element in the array represents the coordinates as a double value of a landmark in a three-dimensional space. 
	The array is used to store and manage the positions of landmarks, which can be loaded from EEPROM, updated via BLE communication, and saved back to EEPROM. 
	
\end{itemize}

\subsection{BleConfigLoader Construktor}
\label{sub:BleConfigLoader_Konstruktor}
The BleConfigLoader creates an instance of the BleConfigLoader class and calls the init\_server method of the my\_server object of type BleServer. 
The method is cescribed in \ref{sub:init_server}. 

\subsection{void save\_config\_to\_eeprom()}
\label{sub:save_config_to_eeprom}
The save\_config\_to\_eeprom method is responsible for storing configuration settings, particularly landmark coordinates, into the EEPROM. 
It iterates through the array of landmark coordinates, calculates EEPROM addresses for each coordinate, and writes the X, Y, and Z values separately to the allocated addresses. 
This function enables the preservation of configuration data across power cycles, ensuring that landmark positions persist even when the device is restarted.

\subsection{void load\_config\_from\_eeprom()}
\label{sub:load_config_from_eeprom}
The load\_config\_from\_eeprom method retrieves configuration settings, specifically landmark coordinates, from the EEPROM. 
It iterates through the array of landmarks, calculates EEPROM addresses for each coordinate that are all next to each other, and reads the X, Y, and Z values separately. 
The coordinates are saved into the landmarkAddresses array. 

\subsection{void save\_config\_to\_ble()}
\label{sub:save_config_to_ble}
The save\_config\_to\_ble method is responsible for broadcasting the current configuration settings, particularly landmark positions, over Bluetooth. 
It uses the BLE server (my\_server) to create a JSON representation of the landmark coordinates and sends this data to a specific BLE characteristic 
\newline 
(BLE\_CHARAKTERISTIK\_ANCHOR\_POSITIONS\_UUID). 
This function allows the desktop application to display the currently saved configuration in the GUI. 

\subsection{uint8\_t load\_config\_from\_ble()}
\label{sub:load_config_from_ble}
The load\_config\_from\_ble method retrieves configuration settings, specifically landmark positions, from another BLE-enabled device running the configuration GUI. 
It utilizes the BLE server (my\_server) to read data from a BLE characteristic
\newline 
(BLE\_CHARAKTERISTIK\_DEVICE\_POSITION\_UUID) and interprets the received JSON-formatted data. 
The function updates the local array of landmark coordinates (landmarkAddresses) with the received values, allowing the BLE configuration loader to synchronize its configuration with the data broadcasted by the GUI. 
If a "save\_config" flag is received over BLE, the function returns 1 initiating the saving process and exiting the configuration mode; otherwise, it returns 0.

\subsection{void send\_position(coordinate own\_position)}
\label{sub:send_position}
The send\_position method is responsible for transmitting the coordinates of the device's own position over Bluetooth. 
It uses the BLE server (my\_server) to send the current coordinates in a specific format to a designated BLE characteristic 
\newline 
(BLE\_CHARAKTERISTIK\_OWN\_POSITION\_UUID). 
This function allows the GUI to receive and visualize use the position information broadcasted by the tag.

\subsection{void print\_config()}
\label{sub:print_config}
The print\_config method prints the loaded configuration settings, specifically the positions of landmarks, to the Serial Monitor for debugging purposes. 
It iterates through the array of landmark coordinates, creates a JSON representation for each set of coordinates, and prints the information, including the landmark ID and its corresponding coordinates, to the Serial Monitor. 
