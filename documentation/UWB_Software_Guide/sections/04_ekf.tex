\chapter{MQTT Handling}
\label{chap:MQTT_Handling}

The MqttClient class is designed for managing MQTT communication in an ESP32-based Arduino environment. 
It encapsulates the functionality for establishing and maintaining a connection to an MQTT broker, handling incoming messages through a callback function, and publishing messages to specified MQTT topics. 
The class includes methods for checking the connection status, obtaining the MAC address of the associated WiFi client, and ensuring periodic updates to handle MQTT events.
The necessary information to establish the connection are set in the main file. 

\section{MqttClient Class}
\label{sec:MqttClient_Class}

%\vspace{4pt}
%\newline

\subsection{mqtt Variables}
\label{sub:mqtt_Variables}

\subsubsection{Private Variables}

\begin{itemize}
	\item \textbf{WiFiClient espClient}
	\newline
	The espClient is an instance of the WiFiClient class, which is part of the ESP32 Arduino core libraries. 
	This object is used to manage the WiFi connection on the ESP32 device. 
	The WiFiClient class provides methods for establishing, maintaining, and handling communication over a WiFi network.
	\vspace{4pt}
	\newline
	In the context of the MqttClient class, espClient serves as the WiFi client for the ESP32, allowing the device to connect to a WiFi network. 
	It is then utilized by the PubSubClient (MQTT client) to establish and maintain the MQTT connection over the WiFi network. 
	
	\item \textbf{PubSubClient client}
	\newline
	The client is an instance of the PubSubClient class, which is used to handle MQTT communication on the ESP32. 
	The PubSubClient library provides functionalities for interacting with MQTT brokers, allowing the ESP32 to both publish messages to specific topics and subscribe to topics to receive messages.
	\vspace{4pt}
	\newline
	In the context of the MqttClient class, the client object represents the MQTT client used for communication with an MQTT broker. 
	It is associated with the WiFiClient instance (espClient), which manages the underlying WiFi connection. 
	The client object is responsible for connecting to the MQTT broker, handling incoming messages, and publishing messages to specified topics.
	
	\item \textbf{String dev\_id}
	\newline
	dev\_id is a variable of type String, representing the device ID associated with the MQTT client instance. 
	It is passed as a parameter to the constructor and can be utilized for identification or customization purposes within the MQTT communication. 
	
	\item \textbf{const char* topic}
	\newline
	topic is a parameter representing the MQTT topic to which the client subscribes and publishes messages. 
	It is passed as a parameter to the constructor when creating an instance of the MqttClient. 
	The topic variable is crucial for specifying the communication channel within the MQTT protocol, allowing the client to subscribe to and publish messages on a particular topic within the MQTT broker. 
	
\end{itemize}

\subsection{MqttClient Constructor}
\label{sub:MqttClient_Constructor}
The constructor of the MqttClient class initializes the MQTT client instance by setting up the MQTT server details, callback function, and buffer size using the respective methods of the PubSubClient class. 
It then establishes a WiFi connection using the provided SSID and password. 
Finally, it calls the reconnect method described in \ref{sub:reconnect} to connect successfully. 

\subsection{update}
\label{sub:update}
The update method ensures the continuous functionality of MQTT communication. 
It includes a call to the private reconnect method described in \ref{sub:reconnect}, responsible for establishing a connection to the MQTT broker if not already connected. 
Additionally, the method invokes client.loop() to handle MQTT client events, facilitating the processing of incoming messages and other protocol-related tasks. 
Regularly calling the update method in the main loop of the program maintains the MQTT client's connectivity and responsiveness to incoming events.

\subsection{publish}
\label{sub:publish}
The publish method facilitates the publication of MQTT messages to a specified topic. 
It takes three parameters: the MQTT topic to publish to (topic), the message to be published (msg), and the length of the message (plength). 
When invoked, this method attempts to publish the message to the MQTT broker associated with the client. 
If successful, the message is sent to the specified topic; otherwise, an exception is caught, and an error message is printed to the Serial monitor. 
The publish method allows for the integration of MQTT message transmission in the application, providing a means to communicate data or commands over the MQTT protocol.

\subsection{is\_connected}
\label{sub:is_connected}
The is\_connected method checks whether the MQTT client is currently connected to the broker. 
It returns a boolean value, true if the client is connected and false otherwise. 
To check the connectio it uses the connected method of the private client object. 

\subsection{get\_mac}
\label{sub:get_mac}
The get\_mac method retrieves the MAC address of the associated WiFi client. 
It returns the MAC address as a String. 
This method allows the application to obtain the unique hardware address assigned to the ESP32's WiFi interface. 
It uses the macAddress method of WiFiClass. 

\subsection{reconnect}
\label{sub:reconnect}
The reconnect method handles the process of reconnecting the MQTT client to the broker. 
If the client is not already connected, this method attempts to establish a connection in a loop. 
If the connection fails, it disconnects the client and retries the connection every 10 seconds until a successful connection is made. 
Once connected, it subscribes to the specified MQTT topic. 

\subsection{setup\_wifi}
\label{sub:setup_wifi}
The setup\_wifi method is responsible for connecting the ESP32 to a WiFi network using the provided SSID and password. It initiates the WiFi connection and includes a loop that waits until the ESP32 successfully connects to the specified WiFi network. 
During this process, it prints dots to the Serial monitor to indicate the connection attempt. 
Once connected, it prints the IP address of the ESP32. 


\section{MQTT Functions}
\label{sec:MQTT_Functions}
\subsection{subscribe\_callback}
\label{sub:subscribe_callback}
The subscribe\_callback function is a callback handler for MQTT subscriptions in the provided code. 
It is invoked whenever a message arrives on a subscribed MQTT topic. 
It serves as an interrupt-like routine for processing incoming messages. 
The function prints the received MQTT message payload to the Serial monitor, allowing for custom message parsing logic to be implemented based on the application's requirements.
