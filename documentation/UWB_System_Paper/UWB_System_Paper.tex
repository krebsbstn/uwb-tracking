\documentclass[10pt,a4paper,twocolumn]{article}

\usepackage[ngerman]{babel} 
\usepackage{float}
\usepackage{lipsum}  

\input{pisika.dat}

%  Editorial staff will uncomment the next line
% \input{staff.hed}

%% acronyms and glossary
\usepackage[
    single=true,
    sort=true,
    only-used=true
]{acro}
\DeclareAcronym{UWB}{
    short = UWB,
    long = Ultra-Wideband
}

\DeclareAcronym{TWR}{
    short = TWR,
    long = Two-Way Ranging
}

\DeclareAcronym{EKF}{
    short = EKF,
    long = Extended Kalman Filter
}

\DeclareAcronym{NLOS}{
    short = NLOS,
    long = Non-Line-Of-Sight
}

\DeclareAcronym{RTOS}{
    short = RTOS,
    long = Real-Time Operating System
}

\DeclareAcronym{TOF}{
    short = TOF,
    long = time-of-flight
}


%% Pictures
\renewcommand{\figurename}{Abbildung}

\begin{document}

%--------------------------------------------------------------------------
%  fill in the paper's title, author(s), and corresponding institutions
%--------------------------------------------------------------------------
\providecommand{\ShortAuthorList}[0]{S. Krebs, T. Herter} % use "A.~M.~Surname \textit{et al}." for more than three authors.
\title{Ultra-Wideband (UWB) Positioning System Based on ESP32 and DWM3000 Modules}
\author[1]{Sebastian Krebs}
\author[1]{Tom Herter}
\affil[1]{HTWG Konstanz,\linebreak  Faculty of Electronics und Informationtechnologies,\linebreak Germany}
%\affil[2]{Other Institute, XX University}

\date{\dateline{\today}}

\maketitle
\vspace*{-1.3cm}
%\thispagestyle{titlestyle}

\section*{}
%---------------------------------------------------------------------------
%               Include abstract and keywords here
%---------------------------------------------------------------------------
\textbf{\textit{Abstract}} - \textbf{
  In this paper, we introduce an innovative \ac{UWB} positioning system
  that leverages six identical custom-designed boards,
  each featuring an \ac{ESP32} microcontroller and integrated with \ac{DWM3000}
  modules from Quorvo.}

  \textbf{This system is capable of achieving precise localization through \ac{TWR}
  measurements between one designated ''Tag'' board and five other ''Anchor'' boards.
  The collected distance measurements are processed by an \ac{EKF} running locally
  on the Tag board, enabling it to determine its own position with high accuracy,
  relying on the fixed positions of the Anchor boards.}

  \textbf{This paper presents a comprehensive overview of the system's architecture,
  the key components, and the remarkable capabilities it offers for accurate
  indoor positioning and tracking applications.}

\vspace*{.28cm}
\keywords{\textbf{RTLS, UWB, TOF, Positioning, Tracking}}

%---------------------------------------------------------------------------
%               the main text of your paper begins here
%---------------------------------------------------------------------------

\section{Introduction}\label{section:intro}
Indoor positioning and tracking have gained significant importance in various domains,
including logistics, healthcare, industrial automation, and smart infrastructure.
Traditional positioning systems often face challenges in terms of accuracy, scalability,
and robustness.
To address these issues, we present an \ac{UWB} positioning system that capitalizes
on advanced hardware and sophisticated algorithms,
resulting in an exceptional level of accuracy and reliability.

Our \ac{UWB} system comprises six identical boards,
all based on the ESP32 microcontroller, a versatile and powerful platform known
for its capabilities in wireless communication and processing.
These boards are equipped with \ac{DWM3000} modules from Quorvo,
renowned for their high-performance UWB capabilities.

One of these boards is designated as the ''Tag'',
responsible for initiating \ac{TWR} measurements with the remaining five ''Anchor'' boards.
The innovative aspect of our system lies in its ability to perform accurate localization
without reliance on external infrastructure or centralized processing.

The heart of our UWB system is the \ac{EKF} implemented locally on the Tag board.
This EKF takes the distance measurements obtained from TWR with the Anchor boards and,
based on their known fixed positions, computes the real-time position of the Tag board
with remarkable precision.
This decentralized approach not only ensures rapid and reliable positioning but also
offers scalability, making it suitable for various applications where real-time location
data is crucial.

In this paper, we will delve into the details of our \ac{UWB} system,
explaining the hardware setup as well as the firmware architecture.

\section{Measurement Priciple}\label{Section:principle}
\lipsum[2-4]
\section{System Architecture}\label{section:system_arch}

Additionally, we are committed to fostering collaboration and knowledge-sharing
within the community.
Therefore, we have made the entire source code, along with all PCB design files,
readily accessible to the public.
You can find these resources, along with detailed documentation,
on our project's GitHub repository, available at the following link:
\url{https://github.com/krebsbstn/uwb-tracking}.

\section{Hardware Design}\label{section:hardware}
\lipsum[2-4]
\section{Firmware Architecture}\label{section:firmware}
\lipsum[2-4]
\subsection{TOF-Task}\label{section:firmware-tof}
\lipsum[2-4]
\subsection{EKF-Task}\label{section:firmware-ekf}
\lipsum[2-4]
\section{Test Results}\label{section:tests}
\lipsum[2-4]
\section{Conclusion And Outlook}\label{section:conclusion}
\lipsum[2-4]
% Please use pisikabst.bst. You may your own *.bib file.
\bibliographystyle{pisikabst}
\bibliography{bibliography}

\end{document}